%!TEX root = ../template.tex
%%%%%%%%%%%%%%%%%%%%%%%%%%%%%%%%%%%%%%%%%%%%%%%%%%%%%%%%%%%%%%%%%%%%
%% glossary.tex
%% NOVA thesis document file
%%
%% Glossary definition
%%%%%%%%%%%%%%%%%%%%%%%%%%%%%%%%%%%%%%%%%%%%%%%%%%%%%%%%%%%%%%%%%%%%

\typeout{NT FILE glossary.tex}%

\newglossaryentry{fourvec}{
	name={four-vector}, 
	description={Vector used in special relativity composed of 4 components, one scalar time-like, and three vectorial space-like. These vectors behave in special way, such as their norm being \gls{lorentzinvariance}. Can be written in covariant, $X_{\mu}$, and contravariant form, $X^{\mu}$, with the difference being the sign of the time-like components. Example of a contravariant four vector: $X^{\mu}=(X^0,X^1,X^2,X^3)=(X^0,\vb*{X})$}
}
\newglossaryentry{lorentzinvariance}{
	name={Lorentz invariant},
	description={A Lorentz invariant scalar, obtained, for example, from a \gls{MinkNorm}, does not change when operated by a Lorentz Transformation.}
}

\newglossaryentry{MinkNorm}{
	name={Minkowski norm}, 
	description={Yields the Lorentz Invariant norm for a \gls{fourvec}: $p_{\mu}p^{\mu}$. Equivalent to the dot product of a classical vector.}
}

\newglossaryentry{virtual photons}{
	name={virtual photons},
	description={While in reality, during a Coulomb interaction, 'real' particles are not exchanged, the electromagnetic field is still mediated by photons. This way virtual photons are tools used in order to better represent electromagnetic interactions.}
}

