%!TEX root = ../template.tex
%%%%%%%%%%%%%%%%%%%%%%%%%%%%%%%%%%%%%%%%%%%%%%%%%%%%%%%%%%%%%%%%%%%%
%% appendix2.tex
%% NOVA thesis document file
%%
%% Chapter with example of appendix with a short dummy text
%%%%%%%%%%%%%%%%%%%%%%%%%%%%%%%%%%%%%%%%%%%%%%%%%%%%%%%%%%%%%%%%%%%%

\typeout{NT FILE appendix3.tex}%

\chapter{QED considerations}\label{ap:QED}

It should be apparent by now that studying Atomic systems call not only for relativistic effects and corrections, but also for QED ones.

One of the most famous cases where QED came to light was the discovery of the Lamb Shift~\cite{Lamb1947}, when it was discovered Hydrogen's $2s_{1/2}$ and $2p_{1/2}$ levels were in fact, not degenerate (did not have the same energy), contrary to what was expected from solving Dirac's equation. This difference in energy came to be known as the Lamb Shift, only explained by QED effects.


\section{Self-Energy}

The self-energy represents a particle's  emission and reabsorption of virtual photon, present in the particle's own generated field. This interaction has the most impact in the Lamb Shift effect and when performing energy corrections. One of its \glspl{feynmandiagram} can be seen in Figure~\ref{fig:selfen}.


\section{Vacuum Polarization}

As previously stated, electromagnetic fields, such as the Coulomb field generated by the nucleus, are mediated by virtual photons. These photons can lead to the creation of electron-positron pairs which create screening effects. Pair annihilation will follow, leading to the production of another virtual photon (Figure~\ref{fig:vacpol}).

\begin{figure}[h!]
    \centering
    \begin{subfigure}{0.4\textwidth}
        \begin{tikzpicture}
            \begin{feynman}
                \vertex (a);
                \vertex [right of = a,xshift=-0.5cm] (b);
                \vertex [right of = b] (c);
                \vertex [right of = c] (d);
                \vertex [right of = d,xshift=-0.5cm] (e);
        
                \diagram{
                    (a) -- [fermion, edge label'=$e^-$] (e);
                    %(a) -- [boson, edge label=$\gamma$,momentum'=$k$] (d);
                    (b) -- [boson,edge label = $\gamma$,half left] (d) ;
                };
        
            \end{feynman}
        \end{tikzpicture}
        \caption{Self Energy}
        \label{fig:selfen}
    \end{subfigure}
    \hfill
    \begin{subfigure}{0.4\textwidth}
        \begin{tikzpicture}
            \begin{feynman}
                \vertex (a);
                \vertex [right of = a,xshift=-0cm] (b);
                \vertex [right of = b,xshift=-0.5cm] (c);
                \vertex [right of = c,xshift=-0.5cm] (d);
                \vertex [right of = d,xshift=-0cm] (e);
        
                \diagram{
                    (a) -- [boson, edge label=$\gamma$] (b);
                    (b) -- [fermion, edge label=$e^-$,half left,looseness=1.5] (d);
                    (d) -- [fermion, edge label=$e^+$,half left,looseness=1.5] (b);
                    (d) -- [boson, edge label=$\gamma$] (e);
                };
        
            \end{feynman}
        \end{tikzpicture}
        \caption{Vacuum Polarization}\label{fig:vacpol}
    \end{subfigure}\caption{QED Feynman Diagrams}

    

\end{figure}







