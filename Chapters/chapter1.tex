%!TEX root = ../template.tex
%%%%%%%%%%%%%%%%%%%%%%%%%%%%%%%%%%%%%%%%%%%%%%%%%%%%%%%%%%%%%%%%%%%
%% chapter1.tex
%% NOVA thesis document file
%%
%% Chapter with introduction
%%%%%%%%%%%%%%%%%%%%%%%%%%%%%%%%%%%%%%%%%%%%%%%%%%%%%%%%%%%%%%%%%%%

\typeout{NT FILE chapter1.tex}%

\chapter{Introduction}\label{cha:introduction}

\section{Theoretical Introduction}
\subsection{Characteristic x-rays}

\begin{figure}[h!]
    \centering
    \begin{tikzpicture}[
        >={Stealth[scale=2]},
        photon/.style={decorate,decoration={snake,post length=3mm}}
    ]
        \draw (0,0) -- (3,0) node[right]{Inner shell};
        \draw[fill=blue!50!white] (1,0) circle (0.25) node {$\uparrow$};
        \draw[fill=blue!50!white,dashed,opacity=0.5] (2,0) circle (0.25) node {$\downarrow$};
        \draw[->,photon](3.5,1) node[above right]{$h\nu$>E$_{\text{b}}$} -- (2.25,0.25);
    
        \draw[dashed] (0,3) -- (3,3);
        \draw[dashed,draw opacity=0.5] (0,3.1) -- (3,3.1) node[right]{Continuum};
        \draw[dashed,opacity=0.25] (0,3.2) -- (3,3.2);
        \draw[dashed,opacity=0.1] (0,3.3) -- (3,3.3);
        \draw[dashed,opacity=0.05] (0,3.4) -- (3,3.4);
        \draw[fill=blue!50!white] (2,3.5) circle (0.25) node at (2,4){E$_{\text{k}}$=$h\nu$-E$_{\text{b}}$};
        \draw[->] (2,0.3) -- (2,3.2);
    
        \draw[-Stealth] (-0.5,-1) -- (-0.5,4) node[anchor=south east]{E};
        \draw (-0.4,3) -- (-0.6,3) node[left]{0};
        \draw (-0.4,0) -- (-0.6,0) node[left]{-E$_{\text{b}}$};
    \end{tikzpicture}
    \caption{Photoionization}
\end{figure}


\begin{figure}[h!]
    \centering
    \begin{tikzpicture}[
        >={Stealth[scale=2]},
        photon/.style={decorate,decoration={snake,post length=3mm}}
    ]
        \draw (0,0) -- (3,0) node[right]{Inner shell};
        \draw[fill=blue!50!white] (1,0) circle (0.25) node {$\uparrow$};
        \draw[fill=blue!50!white,dashed,opacity=0.5] (2,0) circle (0.25) node {$\downarrow$};
        \draw[->,photon](3.5,1) node[above right]{$h\nu$=E$_{\text{b}_1}$-E$_{\text{b}_2}$} -- (2.25,0.25);
    
        \draw[dashed] (0,3) -- (3,3);
        \draw[dashed,draw opacity=0.5] (0,3.1) -- (3,3.1) node[right]{Continuum};
        \draw[dashed,opacity=0.25] (0,3.2) -- (3,3.2);
        \draw[dashed,opacity=0.1] (0,3.3) -- (3,3.3);
        \draw[dashed,opacity=0.05] (0,3.4) -- (3,3.4);
        \draw[->] (2,0.3) -- (2,1.7);

        \draw (0,2) -- (3,2) node[right]{Upper level};
        \draw[fill=blue!50!white] (2,2) circle (0.25);



        \draw[-Stealth] (-0.5,-1) -- (-0.5,4) node[anchor=south east]{E};
        \draw (-0.4,3) -- (-0.6,3) node[left]{0};
        \draw (-0.4,0) -- (-0.6,0) node[left]{-E$_{\text{b}_1}$};
        \draw (-0.4,2) -- (-0.6,2) node[left]{-E$_{\text{b}_2*}$};
    \end{tikzpicture}
    \caption{Resonant Photoexcitation}
\end{figure}


\subsection{The Dirac Equation}


\section{State of the Art}

\section{Objective list}

\section{Work Plan}
