%!TEX root = ../template.tex
%%%%%%%%%%%%%%%%%%%%%%%%%%%%%%%%%%%%%%%%%%%%%%%%%%%%%%%%%%%%%%%%%%%%
%% abstrac-en.tex
%% NOVA thesis document file
%%
%% Abstract in English([^%]*)
%%%%%%%%%%%%%%%%%%%%%%%%%%%%%%%%%%%%%%%%%%%%%%%%%%%%%%%%%%%%%%%%%%%%

\typeout{NT FILE abstrac-en.tex}%

Throughout this thesis, we will explore the influence that consequently higher excited states will have on Copper's diagram transition lines and their fluorescence yield. Multiple atomic structure calculations will be performed for different excited states configurations (from an electron in the $4s$ orbital up to n=??). This way, it will be possible to observe the energy shifts and shape changes of Copper's K and L diagram transition lines  as they get closer to the ones from ionized Copper. For this purpose, a high performance computational method, \gls{MCDFGME}, will be used in order to performance the necessary atomic structure calculations and to compute all the transition line's parameters, while also evaluating all the K and L fluorescence yields. 

This study can be of high importance since the theoretical results obtained shall be used in order to fully comprehend and analyze experimental data obtained from a high precision \gls{DCS}, located in Paris, in a synchroton line.

Further similar and more complex studies will be performed by implementing and running a script in the \textit{Oblivion} supercomputer located at the University of Évora.

% Palavras-chave do resumo em Inglês
% \begin{keywords}
% Keyword 1, Keyword 2, Keyword 3, Keyword 4, Keyword 5, Keyword 6, Keyword 7, Keyword 8, Keyword 9
% \end{keywords}
\keywords{
  Excited State \and
  Diagram Lines \and
  Fluorescence Yield \and
  \gls{MCDFGME}\and
  \gls{DCS}\and
  High Performance Computing
}
% //TODO Falar na evolução dos parâmetros da risca em função da energia do feixe