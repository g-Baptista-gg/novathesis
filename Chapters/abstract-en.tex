%!TEX root = ../template.tex
%%%%%%%%%%%%%%%%%%%%%%%%%%%%%%%%%%%%%%%%%%%%%%%%%%%%%%%%%%%%%%%%%%%%
%% abstrac-en.tex
%% NOVA thesis document file
%%
%% Abstract in English([^%]*)
%%%%%%%%%%%%%%%%%%%%%%%%%%%%%%%%%%%%%%%%%%%%%%%%%%%%%%%%%%%%%%%%%%%%

\typeout{NT FILE abstrac-en.tex}%

The work performed on this thesis comes as part of the effort to further understand the highly convoluted structure present on Copper's X-ray emission spectra, where, as with many other transition metals, a skewness can be observed on the K$_{\alpha_{1,2}}$, K$_{\beta}$ and L transition lines. These line originate due to the radiative relaxation of the atom's electronic structure post-ionization of inner shell electrons.
However, it is very likely the observed skewness is due to copper's satellite states' transitions.

Throughout this thesis, a study will be performed for the satellite states formed by the excitation of the inner-shell electrons, where, as opposed of the ionization process, usually considered in X-ray calculations, a photoexcitation process occurred.

 Multiple atomic structure calculations will be performed using the \textit{ab initio} state of the art \gls{MCDFGME} code for different excited states configurations.

 The obtained results will then be used in the analysis of experimental data obtained from a High-Precision \gls{DCS}, located in Paris, using a synchrotron X-ray source.

Due to the complexity of the calculations, the process can become quite hefty in terms of computational power and time. Therefore, further similar and more complex studies will be performed by implementing and running a script in the \textit{Oblivion} supercomputer located at the University of Évora.

% Palavras-chave do resumo em Inglês
% \begin{keywords}
% Keyword 1, Keyword 2, Keyword 3, Keyword 4, Keyword 5, Keyword 6, Keyword 7, Keyword 8, Keyword 9
% \end{keywords}
\keywords{
  Atomic Excitation \and
  X-ray lines \and
  \gls{MCDFGME}\and
  \gls{DCS}\and
  High Performance Computing
}
% //TODO Falar na evolução dos parâmetros da risca em função da energia do feixe